\documentclass{article}


% if you need to pass options to natbib, use, e.g.:
%     \PassOptionsToPackage{numbers, compress}{natbib}
% before loading neurips_2022

% References (added by Sophia)
% https://www.overleaf.com/learn/latex/Bibliography_management_in_LaTeX
% https://en.wikibooks.org/wiki/LaTeX/Bibliography_Management
\usepackage[authordate,bibencoding=auto,strict,backend=biber,natbib]{biblatex-chicago}
\usepackage[english]{babel}
\addbibresource{References.bib}

\DefineBibliographyStrings{english}{%
  andothers = {et al.},
}

% ready for submission
\usepackage[final,nonatbib]{neurips_2022}


% to compile a preprint version, e.g., for submission to arXiv, add add the
% [preprint] option:
%     \usepackage[preprint]{neurips_2022}


% to compile a camera-ready version, add the [final] option, e.g.:
%     \usepackage[final]{neurips_2022}


% to avoid loading the natbib package, add option nonatbib:
%    \usepackage[nonatbib]{neurips_2022}


\usepackage[utf8]{inputenc} % allow utf-8 input
\usepackage[T1]{fontenc}    % use 8-bit T1 fonts
\usepackage{hyperref}       % hyperlinks
\usepackage{url}            % simple URL typesetting
\usepackage{booktabs}       % professional-quality tables
\usepackage{amsfonts}       % blackboard math symbols
\usepackage{nicefrac}       % compact symbols for 1/2, etc.
\usepackage{microtype}      % microtypography
\usepackage{xcolor}         % colors



\title{scVI enables batch-effect and noise correction and highlights differences in HPSC transcriptional profiles of Aplastic Anemia patients and healthy donors}


% The \author macro works with any number of authors. There are two commands
% used to separate the names and addresses of multiple authors: \And and \AND.
%
% Using \And between authors leaves it to LaTeX to determine where to break the
% lines. Using \AND forces a line break at that point. So, if LaTeX puts 3 of 4
% authors names on the first line, and the last on the second line, try using
% \AND instead of \And before the third author name.


\author{%
Yegor Kuznetsov$^{1*}$ \quad Sophia Jannetty$^{1*}$\\
$^1$University of Washington\\
\texttt{yegor@uw.edu}\\
\texttt{jannetty@uw.edu}\\
}


\begin{document}


\maketitle


\begin{abstract}
  We should write this abstract paragraph last
\end{abstract}


\section{Introduction}

Aplastic anemia (AA) is a condition in which a patient\textquotesingle s bone marrow fails to form enough red blood cells, white blood cells, and platelets, resulting in pancytopenia \citep{young_aplastic_2018}.
Pathophysiological mechanisms of this disease include direct damage to bone marrow (commonly from chemotherapy), germ line loss-of-function mutations that interfere with blood-cell precursor DNA repair pathways, and autoimmune attack \citep{young_aplastic_2018}. 

Current options for treating AA fall within three categories, each of which attempts to address pancytopenia in a different way.
The first, bone marrow transplantation, aims to replace failing bone marrow with healthy donor marrow. 
This is a curative treatment, but is limited by the prevalence of graft-versus-host disease and by its reliance on tissue donors \citep{young_aplastic_2018}.
% Furthermore, the survival rate for adults over the age of 40 following bone marrow transplantation is only around 50\% \citep{giammarco_transplant_2018}.
The second, immunosuppression, aims to eliminate immune-mediated AA.
This category includes treatment with antilymphocyte globulin (combined with cyclosporine), which has a mild lymphocyte-depleting effect.
This treatment leads to improved blood production in about 66\% of patients \citep{bacigalupo_how_2017}, however 30\% to 60\% of these patients experience relapse that requires years of continued cyclosporine therapy \citep{scheinberg_activity_2012}.
Despite the success of immunosuppressant treatments, the mechanisms by which the immune system damages bone marrow remain unknown; the strongest evidence for immune-mediation of aplastic anemia is the effectiveness of immunosuppressant treatments \citep{young_aplastic_2018}.
The third treatment category, stem-cell stimulation, aims to promote stem-cell regeneration within the patient directly.
One such treatment involves administrating eltrombopag, a synthetic mimic of thrombopoietin (a hormone that stimulates the production of platelets).
This treatment has been shown in limited clinical trials to improve blood production in 80\% of treated patients, though the mechanism of this treatment remains unknown \citep{young_aplastic_2018}.

Despite the severity of this disease, the mechanisms underlying its onset and the treatment options remain obscure due to limitations in experimental techniques.
It is thought that T cells may target hematopoietic stem and progenitor cells (HSPCs), which are blood cell precursor cells, in patients with immune-mediated AA \citep{tonglin_single-cell_2022}.
However HSPCs are severely depleated in AA patients, making them a difficult target to study \citep{zhu_single-cell_2021}.
Recent studies like \citet{tonglin_single-cell_2022} and \citet{zhu_single-cell_2021} have used single-cell RNA-sequencing (scRNA-seq) to perform differential expression analysis between healthy and AA patient HSPCs.
These studies have thus far used PCA to reduce dataset dimensionality, followed by graph-based clustering and likelihood ratio tests to identify significantly differentially expressed genes between clusters \citep{zhu_single-cell_2021}.
These studies have successfully identified differentially expressed genes between healthy and AA HSPCs.
However, their accounting for noise and batch effects is limited to their initial dimension reduction and this initial dimension reduction assumes a generalized linear model is sufficient for accurately mapping onto the low-dimensional manifold underlying the data.



\subsection{Citations within the text}

Of note is the command \verb+\citet+, which produces citations appropriate for
use in inline text.  For example,
\begin{verbatim}
   \citet{giammarco_transplant_2018} investigated\dots
\end{verbatim}
produces
\begin{quote}
  \citet{giammarco_transplant_2018} investigated\dots
\end{quote}


\subsection{Figures}

\begin{figure}
  \centering
  \fbox{\rule[-.5cm]{0cm}{4cm} \rule[-.5cm]{4cm}{0cm}}
  \caption{Sample figure caption.}
\end{figure}


Note that publication-quality tables \emph{do not contain vertical rules.} We
strongly suggest the use of the \verb+booktabs+ package, which allows for
typesetting high-quality, professional tables:
\begin{center}
  \url{https://www.ctan.org/pkg/booktabs}
\end{center}
This package was used to typeset Table~\ref{sample-table}.


\begin{table}
  \caption{Sample table title}
  \label{sample-table}
  \centering
  \begin{tabular}{lll}
    \toprule
    \multicolumn{2}{c}{Part}                   \\
    \cmidrule(r){1-2}
    Name     & Description     & Size ($\mu$m) \\
    \midrule
    Dendrite & Input terminal  & $\sim$100     \\
    Axon     & Output terminal & $\sim$10      \\
    Soma     & Cell body       & up to $10^6$  \\
    \bottomrule
  \end{tabular}
\end{table}



\subsection{Margins in \LaTeX{}}


Most of the margin problems come from figures positioned by hand using
\verb+\special+ or other commands. We suggest using the command
\verb+\includegraphics+ from the \verb+graphicx+ package. Always specify the
figure width as a multiple of the line width as in the example below:
\begin{verbatim}
   \usepackage[pdftex]{graphicx} ...
   \includegraphics[width=0.8\linewidth]{myfile.pdf}
\end{verbatim}
See Section 4.4 in the graphics bundle documentation
(\url{http://mirrors.ctan.org/macros/latex/required/graphics/grfguide.pdf})


A number of width problems arise when \LaTeX{} cannot properly hyphenate a
line. Please give LaTeX hyphenation hints using the \verb+\-+ command when
necessary.


\begin{ack}

\end{ack}


\section*{References}


References follow the acknowledgments. Use unnumbered first-level heading for
the references. Any choice of citation style is acceptable as long as you are
consistent. It is permissible to reduce the font size to \verb+small+ (9 point)
when listing the references.
Note that the Reference section does not count towards the page limit.
\medskip


{
\small
\printbibliography
}


\appendix

\section{Appendix}


Optionally include extra information (complete proofs, additional experiments and plots) in the appendix.
This section will often be part of the supplemental material.


\end{document}
